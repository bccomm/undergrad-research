\documentclass[12pt,letterpaper]{article}
\usepackage{amsmath,amssymb,amsthm}
\title{Automorphism Groups of Character Tables}

\newtheorem{thm}{Theorem}
\newtheorem{lem}{Lemma}
\newtheorem{cor}{Corrolary}
\newtheorem{todo}{To-do}
\newcommand{\rep}{\mathrm{rep}}
\newcommand{\Id}{\mathrm{Id}}
\begin{document}
\maketitle
Let $\mathcal{CT}$ be a fixed character table for some finite group
$G$ and consider the group of automorphisms of $\mathcal{CT}$ given by
permuting its rows and columns.

\begin{thm}
  Suppose that an automorphism of $\mathcal{CT}$ acts by permuting the
  rows and columns cyclically, where both the row
  and column permutations have the same cycle length. Assume also that
  the rows being permuted are linear characters, that for all $g$ in
  any of the conjugacy classes of the permuted characters, we have
  $\chi_k(g)=0$ for all nonlinear characters $\chi_k$, and that both
  the row and column permutations act trivially on all character values except some
  square block of size $n$, as shown here. 
\[
\begin{matrix}
  1 & \hdotsfor{3}     & 1 & 1 & \cdots & 1\\
 \vdots   & \ddots & &              & e_2 & e_2 & \cdots & e_2\\
 \vdots   & & \ddots &              & \vdots & \vdots & \vdots & \vdots\\
 \vdots   & &  & \ddots             & e_{l} & e_{l} & \cdots & e_{l}\\
1 & d_2 & \cdots & d_{k} & \alpha_1    & \alpha_{n} & \cdots & \alpha_2\\
1 & d_2 & \cdots & d_{k} & \alpha_2    & \alpha_1    & \cdots & \alpha_3\\
\vdots    & & &              & \vdots     & \vdots      & \vdots & \vdots\\
1 & d_2 & \cdots & d_{k} & \alpha_{n} & \alpha_{n-1} & \cdots & \alpha_1\\
\hdotsfor{4}            & 0 & 0 & 0 & 0\\    
  
\end{matrix}
\]

Then $[G:G^{\prime}]=n+1$.
\end{thm}

\begin{proof}
By column orthogonality,
$\sum_{\chi\in\mathrm{Irr}(G)}\chi(g)\overline{\chi(g)} = |C_G(g)|$. If
$g$ is any representative of the permuted columns, each character
value $c$ is a root of unity, so $c\overline{c}=1$ and this sum
evaluates to $l+n$. Hence each of the permuted conjugacy classes has
the same size. Now consider the other inner products among the
permuted columns. This yields the system of equations 

\begin{align*}
l + \sum_{i=1}^n \alpha_{\sigma_1(i)}\overline{\alpha_{\sigma_2(i)}} = 
\begin{cases}
  |C_G(g)| = l+n,& \sigma_1=\sigma_2\\
  0,&\sigma_1\neq\sigma_2\\
\end{cases}
\end{align*}

as $\sigma_1$ and $\sigma_2$ range over the cyclic permutations of
$\{1,2,\ldots,n\}$ and act by permuting the indices of the $\alpha_i$.

Subtracting $l$ from both sides, the equations can be expressed by the
matrices

\begin{align*}
\begin{bmatrix}
\alpha_1    & \alpha_{n} & \hdots & \alpha_2\\
\alpha_2    & \alpha_1    & \hdots & \alpha_3\\
\vdots      & \vdots     & \ddots & \vdots\\
\alpha_{n} & \alpha_{n-1} & \hdots & \alpha_1
\end{bmatrix}
\begin{bmatrix}
\overline{\alpha_1}    & \overline{\alpha_2} & \hdots & \overline{\alpha_{n}}\\
\overline{\alpha_{n}} & \overline{\alpha_1} & \hdots & \overline{\alpha_{n-1}}\\
\vdots      & \vdots   & \ddots & \vdots\\
\overline{\alpha_2}    & \overline{\alpha_3} & \hdots & \overline{\alpha_1} 
\end{bmatrix}
 & = &
\begin{bmatrix}
n    & -l & \hdots & -l\\
-l & n & \hdots & -l\\
\vdots      & \vdots   & \ddots & \vdots\\
-l    & -l & \hdots & n 
\end{bmatrix} \\ & = &
nI-l(J-I).
\end{align*}

The matrix on the right hand side has an eigenvalue
$n-(n-1)l$. Observe that the left hand side is the product of a matrix
and its conjugate transpose. For any such product, if $\lambda$ is an
eigenvalue and $u$ a corresponding eigenvector,
$\lambda\langle u,u\rangle = \langle A^*Au,u\rangle = \langle
Au,Au\rangle\geq 0$.
Since $\langle x,x\rangle\geq 0$ for any $x\in\mathbb{C}$, we must
have $\lambda\geq 0$.

Hence the above matrix has only non-negative eigenvalues. In
particular, $n-(n-1)l = n-nl+l\geq 0$. Since the $\alpha_i$ are not
all equal, we have $l\geq 1$, so $n-nl\geq -1$. On the other hand,
$n\geq 2$, so $2-l\geq 0$. If $l=2$, then $n-nl=-n\geq -1$ gives
$n=1$, a contradiction. Thus $l=1$ and $n+l=[G:G^{\prime}]=n+1$.
\end{proof}

\begin{thm}
  Let $R$ and $S$ be the permutation groups of the rows and columns of
  a character table $\mathcal{CT}$, respectively. Then the natural
  homomorphisms $\theta: \mathrm{Aut}{(\mathcal{CT})}\to R$ and $\phi:
  \mathrm{Aut}{(\mathcal{CT})}\to S$ are injective.
\end{thm}

\begin{proof}
  Let $\pi_1,\pi_2\in\mathrm{Aut}{(\mathcal{CT})}$ and suppose that
  $\theta(\pi_1)=\theta(\pi_2)$. Then the actions of $\pi_1$ and
  $\pi_2$ on the rows of $\mathcal{CT}$ are the same. Let
  $\theta(\pi_1)\cdot \mathcal{CT} = \mathcal{CT}^{\prime}$. Since
  $\pi_1$ and $\pi_2$ are automorphisms,
  $\mathcal{CT}=\phi(\pi_1)\cdot\mathcal{CT}^{\prime}=\phi(\pi_2)\cdot\mathcal{CT}^{\prime}$. Let
  $\mathcal{K}$ be a column of $\mathcal{CT}$. Then there are columns
  $\mathcal{K}_1$ and $\mathcal{K}_2$ such that $\phi(\pi_1)$ maps
  $\mathcal{K}_1$ to $\mathcal{K}$ and $\phi(\pi_2)$ maps
  $\mathcal{K}_2$ to $\mathcal{K}$. But since $\phi\pi_i$ act only on
  columns, it must be that $\mathcal{K}_1=\mathcal{K}_2$. If not,
  there would be two columns, the pre-images of $\phi\pi_i$, that have
  the same entries---a contradiction since $\mathcal{CT}$ is an
  invertible matrix. Hence both the row and column actions of $\pi_1$
  and $\pi_2$ are the same.
\end{proof}


\begin{thm}
  Every column permutation acts trivially on the trivial character,
  and every row permutation acts trivially on the character degrees.
\end{thm}

\begin{proof}
  Since the trivial character is the map $\chi(g)=1$ for all $g\in G$,
  every arrangement of the conjugacy classes in $G$ yields the same
  row of ones. 

  Suppose that some row permutation $\pi_{row}$ acts on the column of
  character degrees $\mathcal{K}_1$ and that there is a $\pi_{col}$
  such that $\pi_{col}\pi_{row}$ is an automorphism. Then there is
  some column of $\mathcal{CT}$ that is an arrangement of the
  character degree values, call it $\mathcal{K}_0$. Now consider the
  inner product of these two columns. Since both are arrangements of
  positive integers, the inner product
  $\sum_{\chi\in \mathrm{Irr}(G)}\chi(g)\overline{\chi(h)}$, where $g$
  and $h$ are class representatives of $K_0$ and $K_1$, respectively,
  must be positive. This contradicts the column orthogonality
  theorem unless $\mathcal{K}_1=\mathcal{K}_0$.
\end{proof}

\begin{cor}
  Let $\pi$ be an automorphism of $\mathcal{CT}$, and $\pi_{row}$ be
  the corresponding row permutation.  Then
  $\pi_{row}\cdot\chi_i(1)=\pi_{row}\cdot\chi_j(1)$ for all characters
  $\chi_i,\chi_j\in\mathrm{Irr}(G)$. That is, for any automorphism of
  $\mathcal{CT}$, the the action of the row permutation maps every
  character to a character of the same degree.
\end{cor}

\begin{cor}
  Every row permutation fixes the trivial character, and every column
  permutation fixes the character degrees.
\end{cor}

\begin{thm}
   Suppose that there is a character table automorphism $\phi$ in a group
   $H\simeq G/\langle z\rangle$, where $G$ is a non-abelian $2$-group
   and $z$ is a central element of order two. Then $\phi$ is also an
   automorphism of $G$. 
\end{thm}

\begin{proof}
  We have two \emph{or more} cases.

   Case 1. Assume that this automrophism affects only classes that, as
   classes in G, stabilize the action of $z$. In other words, if
   $\phi\in\mathrm{Aut}(\mathcal{CT}(H))$ and
   $\phi_{col}(\mathcal{K}_i)=\mathcal{K}_j$, then for all
   $g\in\mathcal{K}_i$, $g\sim_{\tiny G} gz$.

   Let $\chi_S$ be an irreducible character of $G$ that is not an
   $H$-character. Then if $z\in$ ker $\chi_S$, we are
   done. \emph{FIXME: Why?} If it is not, then as a central element of
   order, it must be mapped to $-\chi_S(1)$. Let $\rho_S$ be the
   irreducible representation corresponding to $\chi_S$. Then if $g$
   and $h$ are conjugacy class representatives of the permuted
   columns, we know that $g\cong gz$ and $h\cong hz$. But since
   $\rho_S(z)=-I_{\chi_S(1)}$, $\chi(g)=-\chi(g)$ and likewise for
   $h$, so both character values must be 0. \emph{FIXME: Why doesn't
     this kill every character value?}

   Case 2. $g\nsim gz$
\end{proof}

\begin{todo}
  Let $\pi$ and $\pi_{col}$ be as above. If
  $\pi_{col}$ maps $\mathcal{K}_1$ to $\mathcal{K}_2$, then $|\mathcal{K}_1|=|\mathcal{K}_2|$.
\end{todo}

\emph{This can probably be combined with the next result}

\begin{todo}
  Suppose that $\pi$ is an automorphism of a character table. Then the
  row and column permutations $\theta(\pi)$ and $\phi(\pi)$ have the
  same cycle shape.
\end{todo}

\begin{proof}
  Let $\Phi_r$ and $\Phi_c$ be permutation matrices corresponding to
  the row and column permutations respectively, so that if
  $M=\mathcal{CT}(G)$, $\Phi_r M \Phi_c^T=M$. We then have $\Phi_C=M^T\Phi_R(M^T)^{-1}$.

  \emph{This is not quite finished. We need to show that conjugation
    by $M$ can be rewritten as conjugation by permutation matrices.}
  By column orthogonality relations,
  $M^*M=\mathrm{diag}\{|C_G(g)|\}$. Rewrite this as
  $\Phi_c M^* \Phi_r^T \Phi_r M \Phi_c^T $. Since permutation matrices
  are orthogonal, this simplifies to $\Phi_c M^* M \Phi_c^T$. Since
  column permutations map conjugacy classes to classes of the same
  size, $\Phi_r$ must act trivially on $M^*M$.
\end{proof}

\begin{todo}
  The character table of every non-abelian solvable group except $S_3$
  has a nontrivial automorphism.
\end{todo}



\end{document}