
\documentclass[xcolor=dvipsnames]{beamer}
\usetheme{Madrid}
\setbeamertemplate{blocks}[default]
\setbeamertemplate{items}[square]
\usecolortheme[named=OliveGreen]{structure}
\usepackage{amsmath,amsthm,amssymb,amscd,amsfonts}
\setbeamertemplate{navigation symbols}{}
\usepackage{graphicx,color}
\useoutertheme{infolines}

\newcommand{\vsum}{underset{v}{\oplus}}
\theoremstyle{remark}
\newtheorem{case}{Case}
\newtheorem{subcase}{Subcase}[case]

\theoremstyle{plain}
\newtheorem{thm}{Theorem}
\newtheorem*{lemma*}{Lemma}
\newtheorem{proposition}[theorem]{Proposition}

\theoremstyle{definition}
\newtheorem{observation}[theorem]{Observation}
\newtheorem{ex}{Example}
\newtheorem*{remark}{Remark}
\newtheorem{cor}{Corollary}
\newtheorem{question}[theorem]{Question}
\DeclareMathOperator{\pin}{pin}
\DeclareMathOperator{\mr}{mr}
\DeclareMathOperator{\stc}{sc}
\DeclareMathOperator{\rank}{rank}
\DeclareMathOperator{\mn}{min}
\DeclareMathOperator{\I}{\mathcal I}
\DeclareMathOperator{\cc}{cc}
\DeclareMathOperator{\T}{T}

\begin{document}
\title[Character Table Automorphisms]{Automorphism Groups of Character
Tables}
\author[ ]{Bruce Chiarelli \and\\ mentored by Dr. Stephen Humphries}
\institute[ ]{Brigham Young University}
\date{March 21, 2015}

\begin{frame}
\titlepage
\end{frame}

\begin{frame}
\frametitle{Definitions}
  Let $G$ be a finite group. Fix an ordering of the conjugacy classes
  $(g_1^G,g_2^G,\ldots,g_r^G)$ and irreducible characters
  $(\chi_1,\chi_2,\ldots,\chi_r)$ of $G$. This produces a fixed
  character table $\mathcal{CT}=[\chi_i(g_j)]$. \pause
\begin{block}{Definition}
  A character table automorphism of $\mathcal{CT}$ is a function $\pi$
  which acts by permuting the conjugacy classes and irreducible
  characters of $G$ so that $\pi(\mathcal{CT})=\mathcal{CT}$.\pause

  In other words, $\pi$ permutes the rows and columns of
  $\mathcal{CT}$ and leaves the character table values unchanged.
\end{block}
\pause The set of automorphisms of $\mathcal{CT}$ forms a group under
composition.
\end{frame}


\begin{frame}
\frametitle{Example: $D_8$}
\[ 
\begin{array}{|c|ccccc|}\hline
  g & e & a^2 & a & b & ab\\
  |g^G| & 1 & 1 & 2 & 2 & 2\\
%  |C_G(g)| & 8 & 8 & 4 & 4 & 4\\
%  |g| & 1 & 2 & 4 & 2 & 2\\
  \hline
  \chi_1 & 1 & 1 & 1 & \textcolor<3->{green}{1} & \textcolor<3->{green}{1}\\
  \chi_2 & 1 & 1 & 1 & \textcolor<3->{green}{-1}& \textcolor<3->{green}{-1}\\
  \chi_3 & \textcolor<3->{green}{1} & \textcolor<3->{green}{1} & \textcolor<3->{green}{-1} & \textcolor<2->{red}{1} & \textcolor<2->{red}{-1}\\
  \chi_4 & \textcolor<3->{green}{1} & \textcolor<3->{green}{1} & \textcolor<3->{green}{-1} & \textcolor<2->{red}{-1}& \textcolor<2->{red}{1}\\
  \chi_5 & 2 & -2& 0 & \textcolor<3->{green}{0} & \textcolor<3->{green}{0}\\\hline
\end{array}\onslide<4->{\rightarrow
\begin{array}{|c|ccccc|}\hline
  g & e & a^2 & a & \textcolor<4>{red}{ab} & \textcolor<4>{red}{b}\\
  |g^G| & 1 & 1 & 2 & 2 & 2\\
%  |C_G(g)| & 8 & 8 & 4 & 4 & 4\\
%  |g| & 1 & 2 & 4 & 2 & 2\\
  \hline
  \chi_1 & 1 & 1 & 1 & 1 & 1\\
  \chi_2 & 1 & 1 & \textcolor<5->{red}{1} & \textcolor<5->{red}{-1}& \textcolor<5->{red}{-1}\\
  \textcolor<4>{red}{\chi_4} & 1 & 1 & \textcolor<5->{red}{-1} & \textcolor<5->{red}{1} & \textcolor<5->{red}{-1}\\
  \textcolor<4>{red}{\chi_3} & 1 & 1 & \textcolor<5->{red}{-1} &
                                                                \textcolor<5->{red}{-1}& \textcolor<5->{red}{1}\\
  \chi_5 & 2 & -2& 0 & 0 & 0\\\hline
\end{array}
}
\]

\onslide<6>{$\mathrm{Aut}(\mathcal{CT}(D_8))=S_3$}

\end{frame}

\begin{frame}
\frametitle{Basic Properties}
\begin{itemize}
\pause
\item Each automorphism $\pi$ can be decomposed into $\pi_{col}$ and
  $\pi_{row}$.\pause
\item If $\mathcal{CT}$ and $\mathcal{CT}^{\prime}$ are character
  tables of the same group, then $\mathrm{Aut}(\mathcal{CT})\cong\mathrm{Aut}(\mathcal{CT}^{\prime})$.\pause
\item The row corresponding to the trivial character and the column
  corresponding to the character degrees always remain fixed.\pause
\item For any $\pi\in\mathrm{Aut(}\mathcal{CT}(G)\mathrm{)}$, the
  corresponding $\pi_{row}$ must map characters to characters of the
  same degree.\pause
\item For any $\pi\in\mathrm{Aut(}\mathcal{CT}(G)\mathrm{)}$, the
  corresponding $\pi_{col}$ must map conjugacy classes to classes
  of the same size.\pause
\item However, if $\pi_{col}$ maps $g^G\mapsto h^G$, it is not
  necessarily the case that $|g|=|h|$.
\end{itemize}
\end{frame}

\begin{frame}
\frametitle{Brauer's Permutation Lemma}
\begin{block}{Lemma (Brauer 1941, Kov\'acs 1982)}
  Suppose that $M$ and $N$ are similar permutation matrices over a
  field of characteristic zero. Then $M$ and $N$ are permutation
  similar. \pause

If $M$ and $N$ are permutation matrices and conjugate as elements of
$GL(n,\mathbb{C})$ then they are conjugate as elements of $S_n$.
\end{block}
%\pause
%\begin{itemize}
%\item Note that $M$ and $N$ are diagonalizable matrices and their
%  eigenvalues are roots of unity.\pause 
%\item Consider the characteristic polynomials of $M$ and $N$, which
%  are products of cyclotomic polynomials. %FIXME
%\end{itemize}
\end{frame}

\begin{frame}
\frametitle{Permutation similarity}
\begin{itemize}
\item Fix a character table $\mathcal{CT}$ of a finite group $G$. Let
$\pi\in\mathrm{Aut(}\mathcal{CT}(G)\mathrm{)}$ and let $\pi_{row}$ and
$\pi_{col}$ be the corresponding row and column permutations. \pause

\item Let $R$ and $C$ be the permutation matrices corresponding to
$\pi_{row}$ and $\pi_{col}$, respectively. Since $\pi$ is an
automorphism of $\mathcal{CT}$, we can write
$\mathcal{CT}=R[\mathcal{CT}]C^T$.\pause

\item Rearranging, we have $C=[\mathcal{CT}]^{-1}R[\mathcal{CT}]$, so $C$
and $R$ are similar permutation matrices.\pause
\end{itemize}

\begin{block}{Theorem}
  $\pi_{row}$ and
  $\pi_{col}$ have the same cycle shape.
\end{block}
\end{frame}

\begin{frame}
\frametitle{Groups with nontrivial CT Automorphisms}
Which groups have a nontrivial character table automorphism?\pause\vspace{.3cm}

We have found explicit character table automorphisms for
\begin{itemize}
\item Dihedral and generalized dihedral groups
\item Extraspecial 2-groups
\item Generalized quaternion groups\pause
\end{itemize}
\begin{block}{Conjecture}
Every non-abelian solvable group except $S_3$ has a nontrivial character table automorphism.
\end{block}
\end{frame}

\begin{frame}
\frametitle{Character lifting in 2 groups}
Let $|G|=2^n$ and choose $z\in Z(G)$, where $|z|=2$. Let $H=G/\langle
z\rangle$.\pause\vspace{.3cm}

The character table of $H$ is embedded within the character table of
$G$.\pause\vspace{.3cm}

%Can an automorphism of $\mathcal{CT}_H$ be lifted to $\mathcal{CT}_G$
%in the same way we lift characters?\pause

If we lift the characters of $H$ to $G$, under what conditions are
character table automorphisms preserved?\vspace{.3cm}

\[
\begin{array}{ccccccc}
  \mathcal{CT}_H       &   &        &       & \overline{g_1} & \overline{g_2}\\
  \chi_1 & 1 & \cdots &   1 & 1              & 1\\
         &   &   &             & b_1            & b_1\\
         &   &   &             & \vdots         & \vdots\\
         &   &   &             & b_l            & b_l\\
  \chi_s & a_1 &\cdots & a_k & c & d\\
  \chi_t & a_1 &\cdots & a_k & d & c\\
\end{array}
\begin{array}{|ccccccc}
  \mathcal{CT}_G         &   &       &       & g_1 & g_2 & g_G\\
  \chi_1 & 1 & \cdots  & 1 & 1              & 1 & 1\\
         &   &   &             & b_1            & b_1 &\\
         &   &   &             & \vdots         & \vdots&\\
         &   &   &             & b_l            & b_l&\\
  \chi_s & a_1 &\cdots  & a_k & c & d &a_i\\
  \chi_t & a_1 &\cdots  & a_k & d & c &a_i\\
  \chi_u &     &        &    & \mu & \mu^{\prime} &\\ 
\end{array}
\]

\end{frame}

\begin{frame}
\[
\begin{array}{ccccccc}
  \mathcal{CT}_H       &   &        &       & \overline{g_1} & \overline{g_2}\\
  \chi_1 & 1 & \cdots &   1 & 1              & 1\\
         &   &   &             & b_1            & b_1\\
         &   &   &             & \vdots         & \vdots\\
         &   &   &             & b_l            & b_l\\
  \chi_s & a_1 &\cdots & a_k & c & d\\
  \chi_t & a_1 &\cdots & a_k & d & c\\
\end{array}
\begin{array}{|ccccccc}
  \mathcal{CT}_G         &   &       &       & g_1 & g_2 & g_G\\
  \chi_1 & 1 & \cdots  & 1 & 1              & 1 & 1\\
         &   &   &             & b_1            & b_1 &\\
         &   &   &             & \vdots         & \vdots&\\
         &   &   &             & b_l            & b_l&\\
  \chi_s & a_1 &\cdots  & a_k & c & d &a_i\\
  \chi_t & a_1 &\cdots  & a_k & d & c &a_i\\
  \chi_u &     &        &    & \mu & \mu^{\prime} &\\ 
\end{array}
\]
\pause
Assume that $g_1\sim g_1z$ and $g_2\sim g_2z$.\pause\vspace{.2cm}

If $z\in\mathrm{ker}(\chi_u)$, $\chi_u$ is an irreducible character of
$H$ and we are done. Suppose that $z\notin\mathrm{ker}(\chi_u)$.\pause\vspace{.2cm}

Since $z$ is a central element of order 2 not in the kernel of $\chi_u$,
$\rho_u(z)=-I_{\chi_{u}(1)}$.\pause\vspace{.2cm}

$\mu=\chi_u(g_1)=\chi_u(g_1z)=-\chi_u(g_1)=-\mu$, so
$\mu=0$. Similarly $\mu^{\prime}=0$. \vspace{.2cm}

\end{frame}

\begin{frame}
\frametitle{Outer automorphisms}
\begin{itemize}
\item For any group $G$, we have a group of automorphisms
$\mathrm{Aut}(G)$. The normal subgroup of inner automorphisms, denoted
$\mathrm{Inn}(G)$ is defined to be those that act by conjugation $G$.\pause

\item Let $\mathrm{Out}(G)=\mathrm{Aut}(G)/\mathrm{Inn}(G)$.\pause

\item Outer automorphisms act on character tables by permuting conjugacy
classes and characters.\pause

\item If $G$ is a non-trivial $p$-group, then it has a non-trivial outer
automorphism. (Gasch\"utz 1965)\pause

\item There exist groups with outer automorphisms that act trivially
  on conjugacy classes, called \emph{class-preserving outer
    automorphisms}. Denoted $\mathrm{Out}_C(G)$, they form ac7 subgroup
  of $\mathrm{Out}(G)$.\pause

\item If we know that $|\mathrm{Out}_C(G)|\lneq |\mathrm{Out}(G)|$, as we do
for several classes of $p$-groups, we can narrow the search for
counterexamples. 
\end{itemize}
\end{frame}

\begin{frame}
\centering \LARGE Thank you!
\end{frame}


\end{document}




