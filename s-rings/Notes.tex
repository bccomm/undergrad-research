\documentclass[12pt,letterpaper]{article}
\usepackage{amsthm,amssymb}
\newtheorem{theorem}{Theorem}
\begin{document}
We seek to find centralizers for sums of transpositions as group
algebra elements.

If there is only one summand, WOLOG $(12)$ we find immediately that disjoint
elements and elements of the form $(1k)+(2k)$ comprise the
centralizer.

For two summands there are two cases to consider. First we develop a
general method for calculating the centralizers.

\begin{theorem}
The product of symmetric matrices is commutative if and only if the
product is symmetric.
\end{theorem}

We begin by encoding the information into a matrix. Let $M$ be the $(0,1)$
matrix where $M=[m_{ij}=1$ whenever $(ij)$ appears in the summand and
zero otherwise. FIXME: This homomorphism is injective, since we are
only considering transpositions.

We then write the block matrix 
$$ \left[
\begin{array}{cc}
M & 0\\ 
0 & |S_1|I_k\\
\end{array}
\right ]
$$

and demand that it commute with 

$$
\left[
\begin{array}{cc}
A & B \\
B^T & C\\
\end{array}\right] .
$$

We obtain the relations $MA=AM$ and $MB=|S_1|B$. Suppose that 

\[
B= \left(
\begin{array}{c}
\mathbf{b_1}\\
\mathbf{b_2}\\
\vdots\\
\mathbf{b_k}\\
\end{array}
\right)
\]

Then $b_i=b_j$ iff $(ij)$ appears somewhere in the summation.

We find then that $A$ corresponds to trivial commutators (Field
elements of the group algebra and multiples of the original summand),
$B$ corresponds to all nontrivial summands, and $C$ is completely
arbitrary, corresponding to sums of disjoint transpositions.

For the case $(12)+(13)$,
M= \[
\left[
\begin{array}{ccc}
0 & 1 & 1\\
1 & 1 & 0\\
1 & 0 & 1\\
\end{array}\right]
\]

We find that $b_1=b_2=b_3$, meaning that in this portion of the
matrix, all commutators must be of the form $(1a)+(2a)+(3a)$, and
indeed all these elements commute. 

For the case $(12)+(34)$, we find that $b_1=b_2$ and $b_3=b_4$,
meaning that nontrivial commutators must be of the form
$t(1a)+t(2a)+s(3b)+s(4b)$, where $s,t\in\{0,1\}$. 
\end{document}